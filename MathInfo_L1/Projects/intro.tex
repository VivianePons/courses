Le principe du projet est d'étudier une question mathématique complexe de façon autonome et originale en utilisant l'outil informatique. \textbf{L'ensemble du projet est à faire par groupe de 3 étudiants.} Le projet est divisé en 2 sections : théorique et pratique. La première partie est \textbf{obligatoire}, la seconde partie correspond à des \textbf{pistes de travail}.

\subsection*{Rendu du projet}\strut

Vous devrez rendre:

\begin{itemize}

\item un \textbf{rapport de mi-projet} répondant aux questions de la section 1;

\item un \textbf{exposé} en fin de projet présentant votre travail et vos résultats sur la section 2.
\end{itemize}

\subsection*{Le rapport}\strut

\smallskip
\textbf{Date de rendu} : cf site du cours \url{https://www.lri.fr/~pons/teaching-mathinfo-l1.html}

\smallskip
\textbf{Que faut-il faire ?} Répondre aux questions de la section 1.

\smallskip
\textbf{Quel format ?} Format PDF obligatoire. Si vous souhaitez ajouter des dessins, vous pouvez les scanner et les ajouter à votre document.

\smallskip
\textbf{Qui réalise le rapport ?} Les membres du groupe réfléchissent ensemble au rapport mais rédigent chacun leur propre rapport. N'oubliez pas de préciser sur votre document qui sont les autres membres du groupe !

\smallskip
\textbf{Comment le rendre ?} Dans votre projet SageMathCloud, vous trouverez un dossier \og Rapport\fg, c'est là qu'il faut uploader votre rapport PDF.

\subsection*{L'exposé}\strut

\smallskip
\textbf{Quand ?} Les exposés auront lieu courant mai, la date sera précisée ultérieurement.

\smallskip
\textbf{Que doit-on présenter ?} Vous devez présenter le travail effecuté sur la section 2 du projet, en particulier les résultats que vous avez obtenus, les algorithmes que vous avez utilisés, les images ou vidéos produites, etc.

\smallskip
\textbf{En combien de temps ?} Vous aurez 10 minutes de présentation, puis 5 minutes de questions.

\smallskip
\textbf{Sur quel support ?} Vous aurez un vidéo projecteur et un ordinateur à disposition (ou le votre si vous le souhaitez). Vous pourrez donc présenter votre exposé sous forme d'un powerpoint ou pdf. Vous pouvez aussi montrer des images, vidéos, démos de code.

\smallskip
\textbf{Qui parle ?} Tout le monde ! Les 3 étudiants doivent participer.

\smallskip
\textbf{Doit-on présenter notre code ?} Vous pouvez utiliser un notebook SageMathCloud pour présenter des démos de code, cependant nous ne notons pas la programmation mais bien les résultats obtenus !

\smallskip
\textbf{Doit-on répondre aux questions ?} Les question de la section 2 ne sont pas obligatoires, ce sont des pistes de travail, vous pouvez les suivre, ou pas...
