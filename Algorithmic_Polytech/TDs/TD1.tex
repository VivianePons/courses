\documentclass{../cours}
\usepackage{hyperref}
\title{TD1}

\begin{document}
\maketitle

\begin{exercice}
Pour chacun des algorithmes suivants, donnez la valeur de $c$ en fonction de $n$ à la fin de l'algorithme puis la complexité de l'algorithme.

\begin{minipage}[t]{0.48 \textwidth}
\begin{lstlisting}
Algo 1 :
    c <- 0
    Pour i allant de 1 à n:
        c <- c + 1
        
Algo 2 :
    c <- 0
    Pour i allant de 1 à n:
        c <- c + 1
    Pour j allant de 2 à  n+1:
        c <- c + 1
        
Algo 3 :
    c <- 0
    Pour i allant de 1 à n:
        Pour j allant de 1 à n:
            c <- c + 1
            
Algo 4 :
    c <- 0
    Pour i allant de 1 à n:
        Pour j allant de 1 à i:
            c <- c + 1
            
            
            
                       
.
\end{lstlisting}
\end{minipage}
\begin{minipage}[t]{0.48 \textwidth}
\begin{lstlisting}
Algo 5 :
    c <- 0
    Pour i allant de 1 à n:
        Pour j allant de 1 à n:
            c <- c + 1
    Pour k allant de 1 à n:
        c <- c + 1

Algo 6 :
    c <- 0
    Pour i allant de 1 à n:
        c <- c + 1
        Pour j allant de 1 à n:
            c <- c + 1
            
Algo 7 :
    c <- 0
    i <- 1
    Tant que i < n:
        i <- i+2
        c <- c + 1
        
Algo 8 :
    c <- 0
    i <- 1
    Tant que i < n:
        i <- i*2
        c <- c + 1
        
\end{lstlisting}
\end{minipage}
\end{exercice}

\begin{exercice}
On considère des algorithmes de complexité : $\log(n)$, $\sqrt{n}$, $n$, $50*n$, $n\log(n)$, $n^2$, $n^3$, $2^n$.
\begin{enumerate}
\item Calculez chacune de ces fonctions pour les puissances de 10 de $10^1$ à $10^{10}$. Le résultat sera sera donné sous la forme approchée $x.10^k$ (Pour $2^n$ one ne calculera que jusqu'à $n=10^3$).
\item En supposant que l'on dispose d'une machine capable de faire $10^6$ (1 million) d'opérations par seconde, quelle est la taille des problèmes que l'on peut résoudre en une seconde, 1000 secondes (environ 17 minutes), 10 000 secondes (environ 2h47) ?
\end{enumerate}
\end{exercice}

\begin{exercice}
Un nombre premier est un nombre qui n'a pas de diviseurs autre que lui-même et 1. \'Ecrivez un algorithme testant la primalité d'un nombre, la complexité doit être inférieure à $O(n)$ (sous-linéaire).
\end{exercice}

\begin{exercice}
\'Ecrivez un algorithme qui prend en paramètre un tableau d'entiers de taille $n$ \emph{triés} (les valeurs sont dans l'ordre croissant) ainsi qu'un entier $a$ et détermine si $a$ est dans $T$. Quelle est la complexité de votre algorithme ? 
\end{exercice}

\begin{exercice}
Une matrice carrée de taille $n$ est un double tableau de taille $n \times n$. Soient $A$ et $B$ deux matrices de taille $n$ dont on écrit les entrées $a_{i,j} = A[i][j]$ et $b_{i,j} = B[i][j]$ avec $0 \leq i,j < n$. Le produit $A \times B$ est donné par la matrice $C$ dont les entrées sont calculées par la formule

\begin{equation*}
c_{i,j} = \sum_{k=0}^{n-1} a_{i,k}b_{k,j}.
\end{equation*}

\begin{enumerate}
\item \'Ecrivez l'algorithme qui calcule la matrice $C$ en fonction de $A$ et $B$ (on supposera qu'on possède une fonction \lstinline{CreeMatrice(n)} qui crée une matrice carrée de taille $n$ et l'initialise à 0.

\item Quelle est la complexité de votre algorithme ? La fonction \lstinline{CreeMatrice} peut-elle l'influencer ?
\end{enumerate}


\end{exercice}

\end{document}