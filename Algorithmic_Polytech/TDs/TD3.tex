\documentclass{../cours}
\usepackage{hyperref}
\title{TD3 : récursivité}

\begin{document}
\maketitle

\begin{exercice}.
\vspace{-0.5cm}
\begin{enumerate}
\item Donner un algorithme récursif qui affiche les entiers de 1 à $n$.
\item Même chose de $n$ à 1.
\item Puis-je modifier ces algorithmes pour qu'ils rajoutent un caractère final de retour à la ligne à la fin de la liste ?
\end{enumerate}
\end{exercice}

\begin{exercice}
Pour chacun des algorithmes suivant :
\begin{enumerate}
\item Déterminer pour quelles valeurs d'entrée l'algorithme termine
\item Calculer un exemple à la main
\item Expliquer ce que calcule l'algorithme
\end{enumerate}

\begin{lstlisting}
Fonction1
Input : un entier n
Processus :
    Si n == 0:
        Retourner 1
    Retourner Fonction1(n+1)
\end{lstlisting}

\begin{lstlisting}
Fonction2
Input : un entier n
Processus :
    Si n == 0:
        Retourner 0
    Retourner Fonction2(n-1)+n
\end{lstlisting}

\begin{lstlisting}
Fonction3
Input : un entier n
Processus :
    Si n == 0:
        Retourner 0
    Retourner Fonction3(n-1)-n
\end{lstlisting}

\begin{lstlisting}
Fonction4
Input : un entier n
Processus :
    Si n == 0:
        Retourner 0
    Si n < 0:
        Retourner n + Fonction4(-n)
    Retourner n + Fonction4(-n+1)
\end{lstlisting}

\begin{lstlisting}
Fonction5
Input : un entier n
Processus :
    Si n <= 1:
        Retourner 0
    Retourner 1 + Fonction5(n-2)
\end{lstlisting}

\end{exercice}

\begin{exercice}
\label{ex-fibo}
La fonction de fibonacci est définie par :
\begin{align*}
U_0 &= 1 \\
U_1 &= 1 \\
U_n &= U_{n-1} + U_{n-2} \text{ si }n \geq 2.
\end{align*}
Ces premières valeurs sont donc : 1, 1, 2, 3, 5, 8, 13, ...

Donner deux algorithmes, un itératif et un récursif, qui calculent la valeur $n$ de la suite.

\end{exercice}

\begin{exercice}
On rappelle sur un exemple le principe de l'algorithme d'Euclide du calcul du pgcd (plus grand diviseur commun). 

Calcul du pgcd de 2145 et 630 : On commence par effectuer la division euclidienne de 2145 par 630

\begin{equation*}
2145 = 630 * 3 + 255
\end{equation*}

puis on effectue la division de 630 par le reste obtenu 255

\begin{equation*}
630 = 255 * 2 + 120
\end{equation*}

On continue jusqu'à ce que l'on trouve un reste nul.

\begin{align*}
255 &= 120 *2 + 15 \\
120 &= 15 * 8
\end{align*}

Le pgcd est le dernier reste non nul, c'est-à-dire \textbf{15}.

Donner un algorithme récursif qui calcule le pgcd de deux nombres.
\end{exercice}

\begin{exercice}
\label{ex:dicho}
Reprendre l'algorithme de recherche dichotomique dans un tableau trié vu dans le TD1 et donner une version récursive.
\end{exercice}

\begin{exercice}
Donner les complexités des algorithmes des exercices \ref{ex-fibo} et \ref{ex:dicho}.
\end{exercice}

\begin{exercice}
Il existe deux définitions récursives de la fonction puissance :
\begin{align*}
a^b &= \begin{cases}
1 &\text{si }b=0 \\
a \times a^{b-1} &\text{sinon}
\end{cases} \\
a^b &= \begin{cases}
1 &\text{si }b=0 \\
a \times a^{b-1} &\text{si }b\text{ est impair} \\
a^{\frac{b}{2}}a^{\frac{b}{2}} &\text{si }b\text{ est pair}
\end{cases}
\end{align*}

Pour chacune des définitions, donner l'algorithme récursif correspondant ainsi que sa complexité.

\end{exercice}

\end{document}