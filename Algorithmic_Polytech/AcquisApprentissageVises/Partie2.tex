\documentclass{../cours}
\usepackage{hyperref}
\title{Acquis d'apprentissage visés, partie 2}

\begin{document}
\maketitle

Ce document présente la seconde partie de ce dont seront capables les étudiants à la suite de l'ensemble des séances de cours, de TP et du travail personnel associé.

\textbf{Ces acquis seront en particulier tous évalués lors du second partiel au mois de mai. Certains acquis liés à la première partie seront aussi \emph{à nouveau} évaluer}

Des exercices types avec grille d'évaluation sont disponibles dans le dossier \texttt{Entrainement}.

\setcounter{aav}{6}

\begin{aav}[Algo de tris]
Les étudiants pourront \textbf{identifier, implanter, calculer la complexité et donner les principales caractéristiques} des algorithmes de tris de base : tri sélection, tri à bulle, tri par insertion, tri rapide, tri fusion.
\end{aav}

\begin{aav}[Arbres et Tas]
Les étudiants seront capables \textbf{d'exécuter, d'implanter, de calculer la complexité} sur des strucutres de données de type \textbf{arbre} :
arbres en structure récursive;
arbres binaires;
arbres binaires parfait et tas;
arbres binaires de recherche.

\begin{itemize}
\item Niveau 1 : algorithme simple dérivant directement de ceux vus en cours / TD, l'étudiant saura reconnaître le problème et connaîtra la complexité optimale ainsi que l'algorithme de résolution.
\item Niveau 2 : la stratégie à appliquer et / ou la complexité optimale n'est pas évidente, il faut parfois combiner plusieurs stratégies. 
\end{itemize}

Le niveau 1 fait partie des acquis de base et doit être atteint par tous les étudiants en partiel. Le niveau 2 doit être atteint par les étudiants lors des séances de TP tutorées (TP à terminer chez soi sur de longues périodes). En partiel, le niveau 2 est considéré comme non obligatoire : il est nécessaire pour obtenir ``20/20'' mais pas pour valider le partiel.
\end{aav}

\end{document}
 