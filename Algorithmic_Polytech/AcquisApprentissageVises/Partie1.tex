\documentclass{../cours}
\usepackage{hyperref}
\title{Acquis d'apprentissage visés, partie 1}

\begin{document}
\maketitle

Ce document présente ce dont seront capables les étudiants à la suite des 3 premières séances de cours, 2 premières séances de TP et du travail personnel associé. 

\textbf{Ces acquis seront en particulier tous évalués lors du premier partiel au mois de décembre.}

Des exercices types avec grille d'évaluation sont disponibles dans le dossier \texttt{Entrainement}.

\begin{aav}[Calcul de complexité]
\textbf{Les étudiants pourront calculer de façon correcte la complexité d'un algorithme itératif simple donné en pseudo-code qu'ils n’auront jamais vu auparavant.} 
\begin{itemize}
\item "simple" signifie que la complexité peut se déduire directement d'une exécution pas à pas de l'algorithme.
\item La réponse sera donnée en utilisant de façon correcte et précise la notation $O$.
\end{itemize}
\end{aav}

\begin{aav}[Conception d'algorithme]
\textbf{Les étudiants pourront concevoir un algorithme de complexité optimale de quelques lignes pour répondre à un problème donné qu'ils n'auront jamais vu auparavant.} Le type de problème et la stratégie de résolution à mettre en place aura été vue lors des séances d'apprentissage (cours ou TP). 
\begin{itemize}
\item Niveau 1 : reformulation directe d'un problème vu en cours / TP, l'étudiant saura reconnaître le problème et connaîtra la complexité optimale ainsi que l'algorithme de résolution.
\item Niveau 2 : la stratégie à appliquer et / ou la complexité optimale n'est pas évidente, il faut parfois combiner plusieurs stratégies. 
\end{itemize}

Le niveau 1 fait partie des acquis de base et doit être atteint par tous les étudiants en partiel. Le niveau 2 doit être atteint par les étudiants lors des séances de TP tutorées (TP à terminer chez soi sur de longues périodes). En partiel, le niveau 2 est considéré comme non obligatoire : il est nécessaire pour obtenir "20/20" mais pas pour valider le partiel.
\end{aav}

\begin{aav}[Structures de données et complexité associées]
Sur les trois structures de données suivantes : \textbf{tableaux, listes chaînées, ensembles}. Les étudiants seront capables de :
\begin{itemize}
\item \textbf{Donner les complexité des opérations de base.}
\item \textbf{Identifier la structure appropriée en fonction des complexités voulues.}
\item \textbf{Identifier l'opération appropriée (ajout / suppression d'un élément en début / fin) sur une structure donnée pour une complexité donnée.}
\end{itemize}
\end{aav}

\begin{aav}[Listes chaînées]
\textbf{Les étudiants pourront implanter des algorithmes de manipulation  type \emph{recherche}, \emph{ajout}, \emph{suppression} sur une structure de \emph{liste chaînée}}.
\end{aav}

\begin{aav}[Analyser un algorithme récursif]
Face à un algorithme récursif donné de quelques lignes en pseudo code, les étudiants seront capables de :
\begin{itemize}
\item \textbf{Calculer le résultat obtenu ou la non terminaison pour des valeurs d'entrée données.}
\item \textbf{Exprimer à l'aide d'une formule mathématique ou d'une phrase le résultat ou la non terminaison pour une valeur quelconque}.
\item \textbf{Calculer le nombre d'appels récursifs pour des valeurs d'entrée données.}
\item \textbf{Exprimer le nombre d'appels récursifs par une formule récursive pour une valeur quelconque.}
\item \textbf{Calculer la complexité de l'algorithme.}
\end{itemize}

\end{aav}

\begin{aav}[Concevoir une fonction récursive]
\textbf{Les étudiants pourront concevoir un algorithme récursif simple pour répondre à un problème donné et calculer sa complexité.}
\begin{itemize}
\item Niveau 1 : fonction mathématique de base avec une récurrence linéaire simple (exemple~: somme des entiers de 1 à $n$). Ou alors, fonction plus complexe mais la récurrence est donnée aux étudiants.
\item Niveau 2 : récurrence plus complexe et non donnée aux étudiants et / ou génération récursive.
\end{itemize}

Le niveau 1 fait partie des acquis de base et doit être atteint par tous les étudiants en partiel. Le niveau 2 doit être atteint par les étudiants lors des séances de TP tutorées (TP à terminer chez soi sur de longues périodes). En partiel, le niveau 2 est considéré comme non obligatoire : il est nécessaire pour obtenir "20/20" mais pas pour valider le partiel.
\end{aav}

\end{document}
 