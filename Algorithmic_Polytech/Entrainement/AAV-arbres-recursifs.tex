\documentclass{../cours}
\usepackage{hyperref}
\title{Entraînement : Arbres structure récursive}

\begin{document}
\maketitle

Pour tous les exercices, la grille d'évaluation est la suivante.

\subsection*{Application d'algorithmes}~

\begin{tabular}{|l|p{12cm}|}
\hline
A (20) & \small{Valeur des fonctions et interprétation correcte} \\ \hline
C (11) & \small{Valeurs des fonctions correctes mais mauvaise interprétation en général.} \\ \hline
D (8) & \small{Erreurs pour les valeurs dans une des deux fonctions.} \\ \hline
E (1) & \small{Les deux sont fausses.} \\ \hline
\end{tabular}

\subsection*{Conception d'algorithmes}~

\begin{tabular}{|l|p{12cm}|}
\hline
A (20) & \small{L'algorithme répond correctement au problème posé, il est écrit de façon claire et la complexité est optimale.} \\ \hline
B (16) & \small{L'algorithme contient quelques erreurs mais reste globalement juste et la complexité est optimale.} \\ \hline
C (11) & \small{L'algorithme fonctionne uniquement dans certains cas ou bien la complexité n'est pas optimale.} \\ \hline
D (8) & \small{L'algorithme ne fonctionne pas.} \\ \hline
E (1) & \small{Algorithme quasi inexistant ou ne répondant pas du tout au problème posé.} \\ \hline
\end{tabular}

Les questions $\clubsuit$ sont ``bonus'' : on ne mettra que $A$, $B$ ou $C$, il faut les réussir pour avoir 20 mais elles n'affectent pas la validation du partiel.


\begin{exercice}

On définit la structure suivante pour représenter les arbres :

\begin{lstlisting}
Structure Arbre :
    valeur, un entier
    nbFils, un entier
    Fils, un tableau de taille nbFils contenant des Arbres
\end{lstlisting}

Et les trois exemples suivants,

\begin{tabular}{c|c|c}
{ \newcommand{\nodea}{\node[draw,circle] (a) {$1$}
;}\newcommand{\nodeb}{\node[draw,circle] (b) {$2$}
;}\newcommand{\nodec}{\node[draw,circle] (c) {$2$}
;}\newcommand{\noded}{\node[draw,circle] (d) {$3$}
;}\newcommand{\nodee}{\node[draw,circle] (e) {$1$}
;}\newcommand{\nodef}{\node[draw,circle] (f) {$4$}
;}\newcommand{\nodeg}{\node[draw,circle] (g) {$2$}
;}
\scalebox{0.8}{
\begin{tikzpicture}[auto]
\matrix[column sep=.3cm, row sep=.3cm,ampersand replacement=\&]{
         \&         \&         \& \nodea  \&         \\ 
         \& \nodeb  \&         \& \nodee  \& \nodef  \\ 
 \nodec  \&         \& \noded  \&         \& \nodeg  \\
};

\path[ultra thick, red] (b) edge (c) edge (d)
	(f) edge (g)
	(a) edge (b) edge (e) edge (f);
\end{tikzpicture}}
}
&
{ \newcommand{\nodea}{\node[draw,circle] (a) {$3$}
;}\newcommand{\nodeb}{\node[draw,circle] (b) {$1$}
;}\newcommand{\nodec}{\node[draw,circle] (c) {$3$}
;}\newcommand{\noded}{\node[draw,circle] (d) {$2$}
;}\newcommand{\nodee}{\node[draw,circle] (e) {$4$}
;}\newcommand{\nodef}{\node[draw,circle] (f) {$6$}
;}\newcommand{\nodeg}{\node[draw,circle] (g) {$1$}
;}\newcommand{\nodeh}{\node[draw,circle] (h) {$1$}
;}\newcommand{\nodei}{\node[draw,circle] (i) {$3$}
;}
\scalebox{0.8}{
\begin{tikzpicture}[auto]
\matrix[column sep=.3cm, row sep=.3cm,ampersand replacement=\&]{
         \& \nodea  \&         \&         \&         \\ 
 \nodeb  \&         \&         \& \nodee  \&         \\ 
 \nodec  \&         \& \nodef  \& \nodeg  \& \nodei  \\ 
 \noded  \&         \&         \& \nodeh  \&         \\
};

\path[ultra thick, red] (c) edge (d)
	(b) edge (c)
	(g) edge (h)
	(e) edge (f) edge (g) edge (i)
	(a) edge (b) edge (e);
\end{tikzpicture}}}
&
{ \newcommand{\nodea}{\node[draw,circle] (a) {$2$}
;}\newcommand{\nodeb}{\node[draw,circle] (b) {$3$}
;}\newcommand{\nodec}{\node[draw,circle] (c) {$1$}
;}\newcommand{\noded}{\node[draw,circle] (d) {$3$}
;}\newcommand{\nodee}{\node[draw,circle] (e) {$5$}
;}\newcommand{\nodef}{\node[draw,circle] (f) {$5$}
;}\newcommand{\nodeg}{\node[draw,circle] (g) {$2$}
;}\newcommand{\nodeh}{\node[draw,circle] (h) {$2$}
;}\newcommand{\nodei}{\node[draw,circle] (i) {$4$}
;}\newcommand{\nodej}{\node[draw,circle] (j) {$3$}
;}
\scalebox{0.8}{
\begin{tikzpicture}[auto]
\matrix[column sep=.3cm, row sep=.3cm,ampersand replacement=\&]{
         \&         \&         \& \nodea  \&         \&         \&         \\ 
         \& \nodeb  \&         \& \nodee  \&         \& \nodeg  \&         \\ 
 \nodec  \&         \& \noded  \& \nodef  \& \nodeh  \& \nodei  \& \nodej  \\
};

\path[ultra thick, red] (b) edge (c) edge (d)
	(e) edge (f)
	(g) edge (h) edge (i) edge (j)
	(a) edge (b) edge (e) edge (g);
\end{tikzpicture}}}
\\
Exemple 1 & Exemple 2 & Exemple 3

\end{tabular}


\begin{enumerate}

\item Donner les valeurs de {\tt mystere1} et {\tt mystere2} sur les 3 exemples d'arbres donnés.
\begin{lstlisting}

Fonction mystere1(Arbre arbre):
    s = 1
    Pour chaque fils f de arbre:
        s = s + mystere1(f)
    retourner s

Fonction mystere2(Arbre arbre):
    m = 0
    Pour chaque fils f de arbre:
        h = mystere2(e)
        Si h > m:
            m = h
    retourner m + 1
Fin fonction

\end{lstlisting}

\item Que calculent {\tt mystere1} et {\tt mystere2} ?

\item Donner une fonction qui calcule le nombre de \emph{feuilles} de l'arbre (nombre de nœuds qui n'ont pas de fils). La fonction doit retourner 4 sur l'exemple 1, 4 sur l'exemple 2 et 6 sur l'exemple 3.


\item $\clubsuit$ On dit qu'un arbre est équilibré si toutes les feuilles sont à la même profondeur. Dans les arbres donnés en exemple, seul l'exemple 3 est équilibré (toutes ses feuilles sont à la profondeur 3, pour les exemples 1 et 2, il y a des feuilles aux profondeurs 2 et 3). Proposer un algorithme qui teste si un arbre donné est équilibré. (Note : il y a plusieurs solutions possibles, plus l'algorithme est optimal en terme de complexité, mieux c'est).

\end{enumerate}


\end{exercice}

\end{document}
