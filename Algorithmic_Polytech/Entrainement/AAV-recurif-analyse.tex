\documentclass{../cours}
\usepackage{hyperref}
\title{Entraînement : analyse d'un algorithme récursif}

\begin{document}
\maketitle
Pour tous les exercices, la grille d'évaluation est la suivante.

\begin{tabular}{|l|p{12cm}|}
\hline
A (20) & \small{Toutes les réponses sont correctes et précises.} \\ \hline
B (16) & \small{Il y a quelques imprécisions ou bien les cas de terminaisons n'ont pas été donnés.} \\ \hline
C (11) & \small{La complexité et le calcul sur des valeurs données sont correctes. } \\ \hline
D (8) & \small{La complexité ou le calcul sur des valeurs données sont correctes. } \\ \hline
E (1) & \small{Ni la complexité, ni le calcul particulier n'est correct.} \\ \hline
\end{tabular}

\begin{exercice}
Voici un algorithme récursif

\begin{lstlisting}
MonCalcul
Input :
    - n, un entier
Procédé :
    Si n = 1:
        Retourner 0
    Retourner 1 + MonCalcul(n/2)
\end{lstlisting}

\begin{enumerate}
\item Calculer la valeur retournée pour les entrées : 1, 2, 3, 4, 5.
\item Donner une valeur d'entrée telle que le résultat soit 4.
\item Sur quelles valeurs d'entrée cet algorithme termine-t-il et que calcule-t-il de façon générale ?
\item Combien d'appels récursifs sont effectués pour la valeur d'entrée 10.
\item Exprimer le nombre d'appels récursives sous forme d'une fonction mathématiques récursives.
\item Donner la complexité de la fonction.
\end{enumerate}

\textbf{Solution}

\begin{enumerate}
\item 1 -> 0 -- 2 -> 1 -- 3 -> 1 -- 4 -> 2 -- 5 -> 2
\item 32
\item Termine sur les entiers supérieurs ou égaux à 1 et calcule Le logarithme en base 2 (partie entière).
\item 4
\item $f(n) = 1 + f(n/2)$ et $f(1) = 1$ 
\item $O(\log(n))$
\end{enumerate}
\end{exercice}


\begin{exercice}[Partiel 2017-18]
Voici un algorithme récursif

\begin{lstlisting}
MonCalcul
Input :
    - n, un entier
Procédé :
    Si n = 0:
        Retourner 1
    Retourner 2*MonCalcul(n-1)
\end{lstlisting}

\begin{enumerate}
\item Calculer la valeur retournée pour les entrées : 1, 2, 3.
\item Sur quelles valeurs d'entrée cet algorithme termine-t-il et que calcule-t-il de façon générale ?
\item Combien d'appels récursifs sont effectués pour la valeur d'entrée 10.
\item Exprimer le nombre d'appels récursives sous forme d'une fonction mathématiques récursives.
\item Donner la complexité de la fonction.
\end{enumerate}

\textbf{Solution}

\begin{enumerate}
\item 1 -> 2 -- 2 -> 4 -- 3 -> 8
\item Termine si $n \geq 0$ et calcule $2^n$.
\item 11 (ou 10 si on ne compte pas le premier appel)
\item $f(n) = f(n-1) + 1$ et $f(0) = 1$
\item $O(n)$
\end{enumerate}

Question donnée au partiel 1 2017-2018, résultats obtenus :

\vspace{0.5cm}

\begin{tabular}{|c|c|c|c|c|}
\hline
A & B & C & D & E \\ \hline
$42.9\%$ & $57.1\%$ & $0\%$ & $0\%$ & $0\%$ \\ \hline
\end{tabular} 
\end{exercice}


\begin{exercice}[Partiel 2018-2019]
Voici un algorithme récursif

\begin{lstlisting}
MonCalcul
Input :
    - n, un entier
Procédé :
    Si n = 0 ou n = 1:
        Retourner 1
    Retourner MonCalcul(n-1) + MonCalcul(n-2)
\end{lstlisting}

\begin{enumerate}
\item Calculer la valeur retournée pour les entrées : 1, 2, 3, 4.
\item Sur quelles valeurs d'entrée cet algorithme termine-t-il ? (On ne demande pas les cas d'arrêt mais toutes les valeurs d'entrée où l'on obtient une réponse).
\item Combien d'appels récursifs sont effectués pour la valeur d'entrée 5.
\item Exprimer le nombre d'appels récursifs sous forme d'une fonction mathématique récursive.
\item Donner la complexité de la fonction.
\end{enumerate}

\textbf{Solution}

\begin{enumerate}
\item 1 -> 1 -- 2 -> 2 -- 3 -> 3 -- 4 -> 5
\item Termine si $n \geq 0$.
\item 15
\item $f(n) = f(n-1) + f(n-2)$
\item $O(2^n)$
\end{enumerate}

Question donnée au partiel 1 2018-2019, résultats obtenus :

\vspace{0.5cm}

\begin{tabular}{|c|c|c|c|c|}
\hline
A & B & C & D & E \\ \hline
$45\%$ & $14\%$ & $5\%$ & $36\%$ & $0\%$ \\ \hline
\end{tabular} 

\end{exercice}


\begin{exercice}[2019 -- 2020]
Voici un algorithme récursif

\begin{lstlisting}
MonCalcul
Input :
    - a, b, 2 entiers
Procédé :
    Si a = 0 ou si b = 0:
        Renvoyer 0
    Si a < 0:
        Renvoyer MonCalcul(-a,b)
    Si a > b:
        Renvoyer MonCalcul(b,a)
    Si a%2 = 0:
        c <- MonCalcul(a/2,b)
        Renvoyer c+c
    Sinon :
        c <- MonCalcul((a-1)/2,b)
        Renvoyer b + c + c
\end{lstlisting}

\begin{enumerate}
\item Calculer la valeur retournée pour les entrées : $a = 8$ ,$b = 10$
\item Que calcule cet algorithme de façon générale ?
\item Sur quelles valeurs d'entrée cet algorithme termine-t-il ? (On ne demande pas les cas d'arrêt mais toutes les valeurs d'entrée où l'on obtient une réponse).
\item Combien d'appels récursifs sont effectués pour les valeurs $a = 8$ ,$b = 10$
\item Exprimer le nombre d'appels récursifs sous forme d'une fonction mathématique récursive.
\item Donner la complexité de la fonction.
\end{enumerate}

\begin{correction}
\begin{enumerate}
\item 80
\item $a \times b$
\item Termine pour tout $a,b$
\item 5
\item $f(a) = 1 + f(\frac{a}{2})$ 
\item $O(\log(n))$ où $n = min(a,b)$
\end{enumerate}
\end{correction}

Question donnée au partiel 2019 -- 2020, résultats obtenus :

\begin{tabular}{|c|c|c|c|c|}
\hline
A & B & C & D & E \\ \hline
41\% & 4\% & 33\% & 11\% & 11\% \\ \hline
\end{tabular} 

\end{exercice}
\end{document}