\begin{exercice}

On définit la structure suivante pour représenter les arbres :

\begin{lstlisting}
Structure Arbre :
    valeur, un entier
    nbFils, un entier
    Fils, un tableau de taille nbFils contenant des Arbres
\end{lstlisting}

Et les trois exemples suivants,

\begin{tabular}{c|c|c}
{ \newcommand{\nodea}{\node[draw,circle] (a) {$1$}
;}\newcommand{\nodeb}{\node[draw,circle] (b) {$2$}
;}\newcommand{\nodec}{\node[draw,circle] (c) {$2$}
;}\newcommand{\noded}{\node[draw,circle] (d) {$3$}
;}\newcommand{\nodee}{\node[draw,circle] (e) {$1$}
;}\newcommand{\nodef}{\node[draw,circle] (f) {$4$}
;}\newcommand{\nodeg}{\node[draw,circle] (g) {$2$}
;}
\scalebox{0.8}{
\begin{tikzpicture}[auto]
\matrix[column sep=.3cm, row sep=.3cm,ampersand replacement=\&]{
         \&         \&         \& \nodea  \&         \\ 
         \& \nodeb  \&         \& \nodee  \& \nodef  \\ 
 \nodec  \&         \& \noded  \&         \& \nodeg  \\
};

\path[ultra thick, red] (b) edge (c) edge (d)
	(f) edge (g)
	(a) edge (b) edge (e) edge (f);
\end{tikzpicture}}
}
&
{ \newcommand{\nodea}{\node[draw,circle] (a) {$3$}
;}\newcommand{\nodeb}{\node[draw,circle] (b) {$1$}
;}\newcommand{\nodec}{\node[draw,circle] (c) {$3$}
;}\newcommand{\noded}{\node[draw,circle] (d) {$2$}
;}\newcommand{\nodee}{\node[draw,circle] (e) {$4$}
;}\newcommand{\nodef}{\node[draw,circle] (f) {$6$}
;}\newcommand{\nodeg}{\node[draw,circle] (g) {$1$}
;}\newcommand{\nodeh}{\node[draw,circle] (h) {$1$}
;}\newcommand{\nodei}{\node[draw,circle] (i) {$3$}
;}
\scalebox{0.8}{
\begin{tikzpicture}[auto]
\matrix[column sep=.3cm, row sep=.3cm,ampersand replacement=\&]{
         \& \nodea  \&         \&         \&         \\ 
 \nodeb  \&         \&         \& \nodee  \&         \\ 
 \nodec  \&         \& \nodef  \& \nodeg  \& \nodei  \\ 
 \noded  \&         \&         \& \nodeh  \&         \\
};

\path[ultra thick, red] (c) edge (d)
	(b) edge (c)
	(g) edge (h)
	(e) edge (f) edge (g) edge (i)
	(a) edge (b) edge (e);
\end{tikzpicture}}}
&
{ \newcommand{\nodea}{\node[draw,circle] (a) {$2$}
;}\newcommand{\nodeb}{\node[draw,circle] (b) {$3$}
;}\newcommand{\nodec}{\node[draw,circle] (c) {$1$}
;}\newcommand{\noded}{\node[draw,circle] (d) {$3$}
;}\newcommand{\nodee}{\node[draw,circle] (e) {$5$}
;}\newcommand{\nodef}{\node[draw,circle] (f) {$5$}
;}\newcommand{\nodeg}{\node[draw,circle] (g) {$2$}
;}\newcommand{\nodeh}{\node[draw,circle] (h) {$2$}
;}\newcommand{\nodei}{\node[draw,circle] (i) {$4$}
;}\newcommand{\nodej}{\node[draw,circle] (j) {$3$}
;}
\scalebox{0.8}{
\begin{tikzpicture}[auto]
\matrix[column sep=.3cm, row sep=.3cm,ampersand replacement=\&]{
         \&         \&         \& \nodea  \&         \&         \&         \\ 
         \& \nodeb  \&         \& \nodee  \&         \& \nodeg  \&         \\ 
 \nodec  \&         \& \noded  \& \nodef  \& \nodeh  \& \nodei  \& \nodej  \\
};

\path[ultra thick, red] (b) edge (c) edge (d)
	(e) edge (f)
	(g) edge (h) edge (i) edge (j)
	(a) edge (b) edge (e) edge (g);
\end{tikzpicture}}}
\\
Exemple 1 & Exemple 2 & Exemple 3

\end{tabular}


\begin{enumerate}

\item Donner les valeurs de {\tt mystere1} et {\tt mystere2} sur les 3 exemples d'arbres donnés.
\begin{lstlisting}

Fonction mystere1(Arbre arbre):
    s = 1
    Pour chaque fils f de arbre:
        s = s + mystere1(f)
    retourner s

Fonction mystere2(Arbre arbre):
    m = 0
    Pour chaque fils f de arbre:
        h = mystere2(e)
        Si h > m:
            m = h
    retourner m + 1
Fin fonction

\end{lstlisting}

\item Que calculent {\tt mystere1} et {\tt mystere2} ?

\item Donner une fonction qui calcule le nombre de \emph{feuilles} de l'arbre (nombre de nœuds qui n'ont pas de fils). La fonction doit retourner 4 sur l'exemple 1, 4 sur l'exemple 2 et 6 sur l'exemple 3.


\item $\clubsuit$ On dit qu'un arbre est équilibré si toutes les feuilles sont à la même profondeur. Dans les arbres donnés en exemple, seul l'exemple 3 est équilibré (toutes ses feuilles sont à la profondeur 3, pour les exemples 1 et 2, il y a des feuilles aux profondeurs 2 et 3). Proposer un algorithme qui teste si un arbre donné est équilibré. (Note : il y a plusieurs solutions possibles, plus l'algorithme est optimal en terme de complexité, mieux c'est).

\end{enumerate}

\textbf{Solution}

\begin{enumerate}
\item mystère 1 : Exemple 1 = 7, Exemple 2 = 9, Exemple 3 = 10. mystère 2 : Exemple 1 = 3, Exemple 2 = 4, Exemple 3 = 3.

\item mystère 1 : nombre de n\oe uds. mystère 2 : hauteur de l'arbre.

\item 

\begin{lstlisting}
Feuilles(Arbre a):
    Si a.nbFils = 0:
        Retourner 1
    s <- 0
    Pour chaque fils f de a:
        s <- s + Feuilles(f)
    Retourner s
\end{lstlisting}


\item Une solution possible (pas la seule), on effectue un parcours en largeur par niveaux

\begin{lstlisting}
Equilibre(Arbre a):
   N1 <- [a] (tableau contenant a)
   Tant que len(N1) > 0:
       N2 <- [] (tableau vide)
       TousFeuilles <- Vrai
       UneFeuille <- Faux
       Pour chaque n dans N1:
          Si n.nbFils = 0:
              UneFeuille <- Vrai
          Sinon :
              TousFeuilles <- Faux
              Pour chaque b dans n.Fils:
                  N2.append(b)
       Si TousFeuilles :
           Retourner Vrai
       Sinon :
           Si UneFeuille :
               RetournerFaux
       
   
\end{lstlisting}

\end{enumerate}


\end{exercice}