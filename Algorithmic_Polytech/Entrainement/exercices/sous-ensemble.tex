
\begin{exercice}[Partiel 2018-2019]
Donner un algorithme qui prend en paramètre un tableau de taille $n$ contenant des valeurs distinctes et renvoie les $2^n$ sous ensemble possible.
Par exemple, si on donne en entrée : {\tt [1,2,3] }, l'algorithme renverra {\tt [ [1,2,3], [1,2], [1,3], [1], [2,3], [2], [3], [] ] }. Dans
chaque tableau, les nombres seront dans l'ordre initial, par contre, l'ordre des sous-ensemble dans le résultat n'a pas d'importance. 

Remarque : pour un tableau vide, on renvoie le tableau contenant l'unique sous-ensemble vide donc {\tt [ [] ] }.

(Vous pouvez utiliser la syntaxe que vous voulez pour tronquer / concaténer des tableaux).


\textbf{Solution}

\vspace{0.5cm}

\begin{lstlisting}
SousEnsemble
Input : un tableau t de taille n
Procédé :
    Si n = 0:
       Renvoyer [[]]
    Resultat <- []
    V <- SousEnsemble(t[1:])
    Pour e dans v:
        Resultat.append([t[0]] + e)
    Resultat.extend(V)
    Renvoyer Resultat
\end{lstlisting}

Question donnée au partiel 1 2018-2019, résultats obtenus :

\begin{tabular}{|c|c|c|c|}
\hline
A & B & C & Non répondu  \\ \hline
$5\%$ & $18\%$ & $50\%$ & $27\%$ \\ \hline
\end{tabular} 
\end{exercice}