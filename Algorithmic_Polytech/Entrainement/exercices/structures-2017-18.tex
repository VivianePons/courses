\begin{exercice}[Partiel 2017-18]~

\begin{enumerate}
\item Quelle est la complexité de l'opération suivante : ajouter un élément au début d'un tableau de taille~$n$.
\item Je cherche une structure telle que je puisse ajouter un élément ou supprimer n'importe quel élément en $O(1)$, que dois-je choisir ?
\item J'ai codé une liste de valeurs par une liste chaînée, je dois rajouter de nombreux éléments. Pour que la complexité soit optimale, dois-je les ajouter en début ou fin de liste ?
\end{enumerate}


\vspace{0.5cm}

\textbf{Solution}


\begin{enumerate}
\item $O(n)$
\item "Ensemble" ou "table de hachage"
\item Début de liste
\end{enumerate}

Question donnée au partiel 1 2017-2018, résultats obtenus :

\begin{tabular}{|c|c|c|c|}
\hline
A & C & D & E \\ \hline
$71.4\%$ & $14.3\%$ & $14.3\%$ & $0\%$ \\ \hline
\end{tabular} 




\end{exercice}