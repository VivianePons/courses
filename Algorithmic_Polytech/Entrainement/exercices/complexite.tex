
\begin{exercice}

Donner la complexité de chacun des algorithmes suivants (sans justification). Chaque algorithme 
prend en entrée un entier $n$. La réponse sera donnée en utilisant la notation $O$.

\begin{minipage}[t]{0.48 \textwidth}
\begin{lstlisting}
Algo_a :
    c <- 0
    Pour i allant de 1 à n:
        Pour j allant de 1 à i:
            c <- c + 1
            
Algo_b :
    c <- 0
    Pour i allant de 1 à n:
        Algo_a(i)
        
Algo_c :
    c <- 0
    Tant que c < n:
        c <- c + 1000
        
        
        
.        
\end{lstlisting}
\end{minipage}
\begin{minipage}[t]{0.48 \textwidth}
\begin{lstlisting}
Algo_d :
    c <- 0
    Tant que c*c < n:
        c <- c+1

Algo_e :
    c <- 1
    Tant que c < n :
        c <- c*2
        
Algo_f :
    f <- 1
    c <- 0
    Tant que c < n:
        i <- 0
        Tant que i < f:
            i <- i+1
        f <- f+i
        c <- c+1
\end{lstlisting}
\end{minipage}

Exemple de réponses et note finale associée. 

Remarque : parfois certaines cases sont laissées vide si la réponse n'a pas été donnée (il vaut mieux avouer
qu'on ne sait pas pas plutôt que d'écrire quelque chose de complètement faux).

\textbf{La première ligne donne les réponses correctes attendues}

\begin{tabular}{|c|c|c|c|c|c|c|}
\hline 
a & b & c & d & e & f & note \\ \hline
$O(n^2)$ & $O(n^3)$ & $O(n)$ & $O(\sqrt{n})$ & $O(\log(n))$ & $O(2^n)$ & A \\ \hline \hline
$O(\frac{n(n-1)}{2})$ & $O(n^3)$ & $O(\frac{n}{1000})$ & $O(\sqrt{n})$ & $O(\log(n))$ & $O(2^n)$ & B \\ \hline
$O(\frac{n(n-1)}{2})$ & $O(n^2)$ & $O(\frac{n}{1000})$ & $\sqrt{n}$ & $\log(\frac{n}{2})$ &  & C \\ \hline
$O(n^2)$ & $O(n^3)$ & $O(n)$ & $O(\frac{n}{2})$ & &  & C \\ \hline
$O(n^2)$ & $O(n^3)$ & $O(n)$ & $O(\sqrt{n})$ & $O(\frac{n}{2})$ &  & D \\ \hline
$O(n^2)$ & $O(n)$ &  & $O(\log(n))$ & $O(\log(n))$ & $O(\log(n))$ & E \\ \hline
\end{tabular}


\end{exercice}