
\begin{exercice}[Partiel 2018-2019]~

\begin{enumerate}
\item Donner un algorithme \textbf{récursif} qui calcule le nombre de chiffre (en base 10) d'un nombre entier positif. On considère que 0 s'écrit avec un chiffre.
Par exemple, pour 16546, on renvoie 5. 

\item Donner la complexité de votre algorithme.
\end{enumerate}

\textbf{Solution}

(Une solution possible)

\begin{lstlisting}
Chiffres
Input : un entier n positif
Procédé :
    Si n < 10:
        renvoyer 1
    renvoyer 1 + Chiffres(n/10)
\end{lstlisting}

Complexité $O(\log(n))$.

Question donnée au partiel 1 2018-2019, résultats obtenus :

\begin{tabular}{|c|c|c|c|c|}
\hline
A & B & C & D & E \\ \hline
$45\%$ & $50\%$ & $5\%$ & $0\%$ & $0\%$ \\ \hline
\end{tabular} 


\end{exercice}