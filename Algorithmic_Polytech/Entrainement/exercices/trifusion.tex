
\begin{exercice}[Tri Fusion]

~
\begin{enumerate}
\item Rappeler en quelques phrases le principe du tri fusion.
\end{enumerate}

En voici une implantation partielle :

\begin{lstlisting}
TriFusion
Input :
    - T, un tableau de taille n 
Output :
    - Un tableau trié contenant les mêmes valeurs que T 
Procédé :
    Si n <= 1:
        Retourner T
    m <- n/2
    T1 <- TriFusion(T[:m+1])
    T2 <- TriFusion(T[m+1:])
    Retourner Fusion(T1,T2)
    
    
Fusion
Input :
    - T1, un tableau de taille m1 supposé trié
    - T2, un tableau de taille m2 supposé trié
Output:
    - un tableau trié contenant les valeurs de T1 et T2
Procédé
    ....
\end{lstlisting}

L'écriture {\tt T[:m+1]} signifie une copie du tableau {\tt T} entre les indices 0 et $m$ inclus, {\tt T[m+1:]} signifie une copie du tableau {\tt T} entre les indices $m+1$ et $n-1$ inclus.

Voici un exemple du résultat de la fonction {\tt Fusion} :


\begin{tabular}{cccccc}
{\tt T1} 2 & 2 & 4 & 5 & 5
\end{tabular}

\begin{tabular}{ccccc}
{\tt T2} 3 & 4 & 6 & 7
\end{tabular}

la fonction renvoie

\begin{tabular}{ccccccccc}
2 & 2 & 3 & 4 & 4 & 5 & 5 & 6 & 7
\end{tabular}

\begin{enumerate}
\setcounter{enumi}{1}

\item Compléter le procédé de la fonction {\tt Fusion}. Votre algorithme doit avoir une complexité $O(n)$ où $n = m1 + m2$.

\item Donner les étapes de votre propre algorithme de fusion sur l'exemple ci-dessus.

\item Quelle est la complexité de l'algorithme complet {\tt TriFusion} dans le pire des cas ?
\end{enumerate}

\textbf{Solution}

\begin{enumerate}
\item L'algorithme découpe le tableau en deux parts égales et trie les deux sous-tableaux puis les fusionne en un tableau unique.
\item cf cours
\item cf cours
\item Pire des cas : $O(n \log(n))$
\end{enumerate}

\end{exercice}