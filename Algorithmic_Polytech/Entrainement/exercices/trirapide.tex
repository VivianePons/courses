
\begin{exercice}[Tri rapide -- 5 pts]

~
\begin{enumerate}
\item Rappeler en quelques phrases le principe du tri rapide (quicksort).
\end{enumerate}

En voici une implantation partielle :

\begin{lstlisting}
TriRapide
Input :
    - T, un tableau de nombres
    - a, le premier indice de la zone à trier
    - b, le dernier indice de la zone à trier 
Procédé :
    Si b <= a:
        Retourner
    m <- Pivot(T,a,b)
    TriRapide(T,a,m-1)
    TriRapide(T,m+1,b)
\end{lstlisting}

Pour un tableau {\tt T} de taille 5, on appellerait la fonction de cette façon {\tt TriRapide(T,0,4)} car 4 est le dernier indice du tableau.

La fonction {\tt Pivot} modifie le tableau de telle sorte qu'après son passage, toutes les valeurs avant l'indice {\tt m} doivent être plus petite ou égale à {\tt T[m]} et toutes les valeurs après l'indice {\tt m} doivent être plus grande que {\tt T[m]}. Par exemple, elle pourrait modifier le tableau suivant de cette façon :

Avant le passage de Pivot :

\begin{tabular}{ccccc}
5 & 2 & 6 & 3 & 1
\end{tabular}

Après le passage de Pivot :

\begin{tabular}{ccccc}
3 & 2 & 1 & \red{5} & 6
\end{tabular}

L'indice {\tt m} retourné par la fonction dans ce cas est \textbf{3}.

\begin{enumerate}
\setcounter{enumi}{1}

\item Donner une implantation de la fonction {\tt Pivot}

\textit{Remarque : } votre fonction n'est pas obligée d'agir exactement comme dans l'exemple, elle doit seulement respecter les propriétés énoncées ci-dessus.

\item Donner les étapes de votre propre algorithme de pivot sur le tableau donné en exemple:

\begin{tabular}{ccccc}
5 & 2 & 6 & 3 & 1
\end{tabular}

\item Quelle est la complexité de l'algorithme {\tt TriRapide} dans les trois cas suivant : tableau déjà trié, meilleur des cas, en moyenne.
\end{enumerate}

\textbf{Solution}

\begin{enumerate}
\item Principe : A chaque étape on choisit un pivot $v$ (par exemple, le premier élément du tableau), puis on place les éléments du tableau tel que tous les éléments \emph{plus petits} que $v$ soient sur la gauche du tableau, tous les éléments \emph{plus grands} sur la droite et $v$ entre la partie gauche et la partie droite. Puis on trie récursivement les parties gauches et droites.

\item cf cours
\item cf cours
\item tableau déjà trié : $O(n^2)$, meilleur des cas : $O(n \log(n))$, en moyenne $O(n \log(n))$.
\end{enumerate}

\end{exercice}