
\begin{exercice}[2019-2020]
On a codé un dictionnaire à l'aide d'un tableau $D$ contenant des mots rangés par ordre alphabétique. 

\begin{enumerate}
\item \textbf{\'Ecrire une fonction qui renvoie l'indice d'un mot donné dans le dictionnaire (ou $-1$ si le mot est absent) de complexité optimale et donner sa complexité.} La fonction prend en paramètre le tableau $D$ de taille $n$ ainsi que le mot $v$. On a accès à une fonction de comparaison {\tt compare(v1,v2)} de complexité $O(1)$ qui prend en paramètre 2 mots et renvoie $-1$ si $v1$ est avant dans l'ordre alphabétique, 0 si les deux mots sont égaux et $1$ et $v1$ est après dans l'ordre alphabétique. 

\item La taille $n$ du dictionnaire correspond au nombre de mots présents. Pour pouvoir rajouter des mots, le dictionnaire a aussi une capacité $c > n$. Lorsqu'on rajoute un mot et que la capacité est atteinte, elle doit être augmentée. Décrire une stratégie d'augmentation de la capacité qui permette d'obtenir une complexité optimale.


\end{enumerate}


\textbf{Solution}

\begin{lstlisting}
Algo1:
Input D tableau de taille n, v un mot
i <- 0
j <- n
Tant que i < j:
    m <- (i+j)//2
    w = D[m]
    c <- 
    Si c = 0:
        Renvoyer m
    Sinon si c < 0:
         j <- m
    Sinon :
         i <- m+1
Renvoyer -1
\end{lstlisting}

Algo2:
On commence avec une capacité donnée (par exemple 1) et on double la capacité quand le dictionnaire est plein.

Complexité de Algo1 : $O(\log(n))$.

Question donnée au partiel 1 2019-2020, résultats obtenus :

\begin{tabular}{|c|c|c|c|c|}
\hline
A & B & C & D & E \\ \hline
41\% & 44\% & 7\% & 4\% & 4\% \\ \hline
\end{tabular} 



\end{exercice}

