
\begin{exercice}
Voici un algorithme récursif

\begin{lstlisting}
MonCalcul
Input :
    - n, un entier
Procédé :
    Si n = 1:
        Retourner 0
    Retourner 1 + MonCalcul(n/2)
\end{lstlisting}

\begin{enumerate}
\item Calculer la valeur retournée pour les entrées : 1, 2, 3, 4, 5.
\item Donner une valeur d'entrée telle que le résultat soit 4.
\item Sur quelles valeurs d'entrée cet algorithme termine-t-il et que calcule-t-il de façon générale ?
\item Combien d'appels récursifs sont effectués pour la valeur d'entrée 10.
\item Exprimer le nombre d'appels récursives sous forme d'une fonction mathématiques récursives.
\item Donner la complexité de la fonction.
\end{enumerate}

\textbf{Solution}

\begin{enumerate}
\item 1 -> 0 -- 2 -> 1 -- 3 -> 1 -- 4 -> 2 -- 5 -> 2
\item 32
\item Termine sur les entiers supérieurs ou égaux à 1 et calcule Le logarithme en base 2 (partie entière).
\item 4
\item $f(n) = 1 + f(n/2)$ et $f(1) = 1$ 
\item $O(\log(n))$
\end{enumerate}
\end{exercice}