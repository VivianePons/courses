
\begin{exercice}[Partiel 2018-2019]

Donner la complexité de chacun des algorithmes suivants (sans justification). Chaque algorithme 
prend en entrée un entier positif $n$. 

\begin{minipage}[t]{0.48 \textwidth}
\begin{lstlisting}
Algo_a :
    c <- 0
    Pour i allant de 1 à n:
        c <- c+2
            
Algo_b :
    c <- 1
    Tant que c < n:
       c <- c*2   
       
Algo_c :
    c <- 0
    Pour i allant de 1 à n:
        c <- c * c
        Pour j allant de 1 à n:
	        c <- c * 2       
\end{lstlisting}
\end{minipage}
\begin{minipage}[t]{0.48 \textwidth}
\begin{lstlisting}
Algo_d :
    c <- 1
    Pour i allant de 1 à n:
        c <- c + 1
        Pour j allant de 1 à i:
            c <- c + 1

Algo_e :
    Tant que n > 0:
        n <- n//2
        
Algo_f :
    c <- 0
    Pour i allant de 1 à n:
        Pour j allant de 1 à i:
            Pour k allant de 1 à j:
                c <- c + 1
\end{lstlisting}
\end{minipage}


Exemple de réponses et note finale associée. 

\textbf{La première ligne donne les réponses correctes attendues}

\begin{tabular}{|c|c|c|c|c|c|c|}
\hline 
a & b & c & d & e & f & note \\ \hline
$O(n)$ & $O(\log(n))$ & $O(n^2)$ & $O(n^2)$ & $O(\log(n))$ & $O(n^3)$ & A \\ \hline \hline
$O(\frac{n}{2})$ & $O(\log(n))$ & $O(n^2)$ & $O(n^2)$ & $O(\log(n))$ & $O(n^3)$ & B \\ \hline
$O(n)$ & $O(\log(n))$ & $O(n^2)$ & $O(n^3)$ & $O(\log(n))$ & $O(n^3)$ & C \\ \hline
$O(n)$ & $O(\log(n))$ & $O(n^2)$ & $O(n^2)$ & $O(\log(n))$ & $O(n^2)$ & C \\ \hline
$O(n)$ & $O(n)$ & $O(n^2)$ & $O(n^2)$ & $O(\log(n))$ & $O(n^3)$ & D \\ \hline
\end{tabular}

Question donnée au partiel 1 2018-2019, résultats obtenus :

\begin{tabular}{|c|c|c|c|c|}
\hline
A & B & C & D & E \\ \hline
60\% & 0\% & 18\% & 22\% & 0\% \\ \hline
\end{tabular} 



\end{exercice}