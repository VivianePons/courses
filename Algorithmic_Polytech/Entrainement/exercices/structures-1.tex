\begin{exercice}
\begin{enumerate}
\item Quelle est la complexité suivante : chercher si un élément $a$ appartient à un ensemble $E$ codé par table de hachage.
\item On cherche à modéliser une liste de valeurs sur laquelle on va devoir faire de nombreux ajouts et suppression en début de liste, 
quelle est la structure appropriée pour optimiser la complexité des opérations ?
\item Une Pile est une structure qui possède deux opérations \emph{push} qui ajoute un élément et \emph{pop} qui supprime le dernier élément ajouté et le retourne. Si je code ma pile par un tableau, ou dois-je ajouter / supprimer les éléments pour que la complexité des opérations \emph{push} et \emph{pop} soit $O(1)$ ? 
\end{enumerate}

\textbf{Solution}

\begin{enumerate}
\item $O(1)$
\item Liste chaînée
\item Fin de tableau.
\end{enumerate}

\end{exercice}