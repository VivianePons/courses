
\begin{exercice}[2019 -- 2020]~

La suite de Syracuse d'un nombre $k$ est donnée par l'algorithme suivant :
\begin{align*}
U_1 &= k \\
U_{n+1} &= \begin{cases}
\frac{U_n}{2}, \text{ si }n\text{ est pair}\\
3U_n + 1, \text{ si }n\text{ est impair}.
\end{cases}
\end{align*}

Par exemple, pour $k = 3$, on obtient $3, 10, 5, 16, 8, 4, 2, 1, 4, 2, 1, \dots$. La conjecture de Syracuse prétend que la suite atteint toujours la valeur $1$. Ainsi, on définit la \emph{taille de Syracuse} $n$ d'un entier $k$ par le plus petit $n$ tel que $U_n = 1$. Pour $k = 3$, la taille est donc 8 car la 8ème valeur de la suite est 1. Pour $7$, on obtient la suite : $7, 22, 11, 34, 17, 52, 26,
13,
40,
20,
10,
5,
16,
8,
4,
2,1$ soit une taille de $17$.

\textbf{\'Ecrire une fonction récursive qui calcule la taille de Syracuse d'un nombre donné}.


\textbf{Solution}

\begin{lstlisting}
Syracuse
Input : un entier k positif
Procédé :
    Si k = 1:
        Renvoyer 1
    Si k%2 = 0:
        Renvoyer 1 + Syracuse(k/2)
    Sinon:
        Renvoyer 1 + Syracuse(3k+1)
\end{lstlisting}



Question donnée au partiel 2019 -- 2020, résultats obtenus :

\begin{tabular}{|c|c|c|c|c|}
\hline
A & B & C & D & E \\ \hline
56\% & 15\% & 15\% & 7\% & 7\% \\ \hline
\end{tabular}
\end{exercice}