
\begin{exercice}[2019 -- 2020]
Donner un algorithme qui prend en paramètre un tableau de taille $n$ contenant des valeurs distinctes et renvoie les $n!$ permutations possible.
Par exemple, si on donne en entrée : {\tt [1,2,3] }, l'algorithme renverra {\tt [ [1,2,3], [1,3,2], [2,1,3], [2,3,1], [3,1,2], [3,2,1]] }. 

Aide : pour un tableau vide, on renvoie le tableau contenant l'unique permutation vide donc {\tt [ [] ] }. Pour un tableau de taille $1$, comme $[1]$, on renvoie l'unique permutation qui est le tableau lui même donc {\tt [ [1] ] }. Quand le tableau est plus grand, une permutation est faite d'un élément du tableau puis d'une permutation des éléments restant.

(Vous pouvez utiliser la syntaxe que vous voulez pour tronquer / concaténer des tableaux et ajouter / supprimer des éléments).

\textbf{Solution}


\begin{lstlisting}
Permutations
Input : un tableau t de taille n
Procédé :
    Si n = 0 ou n = 1:
       Renvoyer [t]
    Resultat <- []
    Pour v dans t:
       t2 = copie(t)
       t2.delete(v)
       Pour p dans Permutations(t2):
           Resultat.append([v] + p)
    Renvoyer Resultat
\end{lstlisting}

Question donnée au partiel 1 2018-2019, résultats obtenus :

\begin{tabular}{|c|c|c|c|}
\hline
A & B & C & Non répondu  \\ \hline
$19\%$ & $33\%$ & $26\%$ & $22\%$ \\ \hline
\end{tabular} 


\end{exercice}