
\begin{exercice}[2019-20]

Donner la complexité de chacun des algorithmes suivants (sans justification). Chaque algorithme 
prend en entrée un entier positif $n$. 

\begin{minipage}[t]{0.48 \textwidth}
\begin{lstlisting}
Algo_a :
    c <- n
    Tant que c > 1:
        c <- c/2 (division entière)
            
Algo_b :
    c <- 1
    Tant que c < n:
       c <- c*2   
       
Algo_c :
    c <- 0
    Pour i allant de 1 à n:
        Pour j allant de 1 à n:
	        c <- c+1   
	        
Algo_d :
    c <- 1
    Pour i allant de 1 à n:
        c <- c + 1
        Pour j allant de 1 à i:
            c <- c + 1
\end{lstlisting}
\end{minipage}
\begin{minipage}[t]{0.48 \textwidth}
\begin{lstlisting}
Algo_e :
    c <- 1
    Pour i allant de 1 à n:
        c <- c + 1
    Pour i allant de 1 à n:
        c <- c + 1
    Pour i allant de 1 à n:
        c <- c + 1
        
Algo_f :
    c <- 1
    Pour i allant de 1 à n:
        c <- c + 1
    Pour i allant de 1 à n:
        c <- c + 1
        Pour i allant de 1 à n:
            c <- c + 1
\end{lstlisting}
\end{minipage}

\textbf{Solution}


\begin{itemize}
\item a -- $O(\log(n))$
\item b -- $O(\log(n))$
\item c -- $O(n^2)$
\item d -- $O(n^2)$
\item e -- $O(n)$
\item f -- $O(n^2)$
\end{itemize}


Question donnée au partiel 1 2019-2020, résultats obtenus :

\begin{tabular}{|c|c|c|c|c|}
\hline
A & B & C & D & E \\ \hline
59\% & 0\% & 26\% & 7.5\% & 7.5\% \\ \hline
\end{tabular} 


\end{exercice}