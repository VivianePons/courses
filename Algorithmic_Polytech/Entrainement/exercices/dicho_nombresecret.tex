
\begin{exercice}[Partiel 2018-2019]
On cherche à évaluer une donnée numérique non entière $x > 1$ par un entier (le plus grand entier $n$ tel que $n < x$). On ne peut pas lire
directement la donnée, on n'a seulement accès à la fonction suivante :

\begin{lstlisting}
infX(k) : 
Renvoie True si k est strictement inférieur à la donnée x et False sinon.
\end{lstlisting}


\begin{enumerate}
\item \textbf{Premier cas} : on sait que $x$ est compris entre deux entiers $v1 < x < v2$. \'Ecrivez un algorithme
optimal qui prend en paramètre $v1$ et $v2$ et renvoie $k$, l'estimation entière de $x$. \textbf{Donnez sa complexité.} (On suppose
que {\tt infX} a une complexité $O(1)$).

\item \textbf{Deuxième cas} : on ne sait rien de la donnée $x$.   \'Ecrivez un algorithme
optimal qui renvoie $k$, l'estimation entière de $x$. \textbf{Donnez sa complexité.}
\end{enumerate}

\textbf{Remarque :} vous avez le droit d'utiliser la première question pour résoudre la deuxième.

\textbf{Solution}


\begin{lstlisting}
Algo1:
Input v1, v2
k <- (v1 + v2) /2
Tant que infX(k) == infX(k+1):
    Si infX(k):
        v1 <- k+1
    Sinon :
        v2 <- k
    k <- (v1 + v2) /2
Renvoyer k

Algo2:
k <- 1
Tant que infX(k):
    k <- k*2
Renvoyer Algo1(k/2,K)
\end{lstlisting}

Complexité des 2 algorithmes : $O(\log(x))$.

Question donnée au partiel 1 2018-2019, résultats obtenus :

\begin{tabular}{|c|c|c|c|c|}
\hline
A & B & C & D & E \\ \hline
18\% & 27\% & 40\% & 13\% & 0\% \\ \hline
\end{tabular}

\end{exercice}

