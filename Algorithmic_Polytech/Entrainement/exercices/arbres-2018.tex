\begin{exercice}[Partiel 2018]

On définit la structure suivante pour représenter les arbres :

\begin{lstlisting}
Structure Arbre :
    valeur, un entier
    nbFils, un entier
    Fils, un tableau de taille nbFils contenant des Arbres
\end{lstlisting}

Et les trois exemples suivants,

\begin{tabular}{c|c|c}
{ \newcommand{\nodea}{\node[draw,circle] (a) {$1$}
;}\newcommand{\nodeb}{\node[draw,circle] (b) {$2$}
;}\newcommand{\nodec}{\node[draw,circle] (c) {$2$}
;}\newcommand{\noded}{\node[draw,circle] (d) {$3$}
;}\newcommand{\nodee}{\node[draw,circle] (e) {$1$}
;}\newcommand{\nodef}{\node[draw,circle] (f) {$4$}
;}\newcommand{\nodeg}{\node[draw,circle] (g) {$2$}
;}
\scalebox{0.8}{
\begin{tikzpicture}[auto]
\matrix[column sep=.3cm, row sep=.3cm,ampersand replacement=\&]{
         \&         \&         \& \nodea  \&         \\ 
         \& \nodeb  \&         \& \nodee  \& \nodef  \\ 
 \nodec  \&         \& \noded  \&         \& \nodeg  \\
};

\path[ultra thick, red] (b) edge (c) edge (d)
	(f) edge (g)
	(a) edge (b) edge (e) edge (f);
\end{tikzpicture}}
}
&
{ \newcommand{\nodea}{\node[draw,circle] (a) {$3$}
;}\newcommand{\nodeb}{\node[draw,circle] (b) {$1$}
;}\newcommand{\nodec}{\node[draw,circle] (c) {$3$}
;}\newcommand{\noded}{\node[draw,circle] (d) {$2$}
;}\newcommand{\nodee}{\node[draw,circle] (e) {$4$}
;}\newcommand{\nodef}{\node[draw,circle] (f) {$6$}
;}\newcommand{\nodeg}{\node[draw,circle] (g) {$1$}
;}\newcommand{\nodeh}{\node[draw,circle] (h) {$1$}
;}\newcommand{\nodei}{\node[draw,circle] (i) {$3$}
;}
\scalebox{0.8}{
\begin{tikzpicture}[auto]
\matrix[column sep=.3cm, row sep=.3cm,ampersand replacement=\&]{
         \& \nodea  \&         \&         \&         \\ 
 \nodeb  \&         \&         \& \nodee  \&         \\ 
 \nodec  \&         \& \nodef  \& \nodeg  \& \nodei  \\ 
 \noded  \&         \&         \& \nodeh  \&         \\
};

\path[ultra thick, red] (c) edge (d)
	(b) edge (c)
	(g) edge (h)
	(e) edge (f) edge (g) edge (i)
	(a) edge (b) edge (e);
\end{tikzpicture}}}
&
{ \newcommand{\nodea}{\node[draw,circle] (a) {$2$}
;}\newcommand{\nodeb}{\node[draw,circle] (b) {$3$}
;}\newcommand{\nodec}{\node[draw,circle] (c) {$1$}
;}\newcommand{\noded}{\node[draw,circle] (d) {$3$}
;}\newcommand{\nodee}{\node[draw,circle] (e) {$5$}
;}\newcommand{\nodef}{\node[draw,circle] (f) {$5$}
;}\newcommand{\nodeg}{\node[draw,circle] (g) {$2$}
;}\newcommand{\nodeh}{\node[draw,circle] (h) {$2$}
;}\newcommand{\nodei}{\node[draw,circle] (i) {$4$}
;}\newcommand{\nodej}{\node[draw,circle] (j) {$3$}
;}
\scalebox{0.8}{
\begin{tikzpicture}[auto]
\matrix[column sep=.3cm, row sep=.3cm,ampersand replacement=\&]{
         \&         \&         \& \nodea  \&         \&         \&         \\ 
         \& \nodeb  \&         \& \nodee  \&         \& \nodeg  \&         \\ 
 \nodec  \&         \& \noded  \& \nodef  \& \nodeh  \& \nodei  \& \nodej  \\
};

\path[ultra thick, red] (b) edge (c) edge (d)
	(e) edge (f)
	(g) edge (h) edge (i) edge (j)
	(a) edge (b) edge (e) edge (g);
\end{tikzpicture}}}
\\
Exemple 1 & Exemple 2 & Exemple 3

\end{tabular}


\begin{enumerate}

\item Donner les valeurs de {\tt mystere1} et {\tt mystere2} sur les 3 exemples d'arbres donnés.

\begin{lstlisting}

Fonction mystere1(Arbre arbre):
    Si arbre.nbFils = 0:
        Retourner 1
    s = 0
    Pour chaque fils f de arbre:
        s = s + mystere1(f)
    retourner s

Fonction mystere2(Arbre arbre):
    Si arbre.nbFils = 0:
        Retourner 0
    s = 1
    Pour chaque fils f de arbre:
        s = s + mystere2(f)
    retourner s

\end{lstlisting}

\item Que calculent {\tt mystere1} et {\tt mystere2} ?

\item On suppose que ces arbres représentent des chemins dans un labyrinthe dont l'entrée est la racine et les feuilles sont les sorties. On veut calculer la \emph{distance minimale} entre la racine et la sortie. Sur les exemples, on obtiendrait la valeur $2$ pour l'exemple 1 (chemin $1 \rightarrow 1$) , et $3$ pour les exemples $2$ et $3$ (chemins $3 \rightarrow 4 \rightarrow 6$ et $2 \rightarrow 3 \rightarrow 1$). \'Ecrire une \textbf{fonction récursive} qui calcule cette distance minimale en utilisant un \textbf{parcours en profondeur}.

\item $\clubsuit$ Imaginez un arbre tel que sa première branche à gauche soit une longue ligne de $10^6$ n\oe uds tandis que sa branche de droite a une profondeur de $10$. Dans ce cas, le parcours en profondeur pour calculer la distance minimale entre la racine et une feuille n'est pas optimal. \'Ecrire une fonction qui calcule cette distance minimale à l'aide d'un parcours en largeur.
\end{enumerate}

\textbf{Solution}

\begin{enumerate}
\item mystere1: 4,4,6 -- mystere2: 3,5,4
\item mystere1: nombre de feuilles -- mystere2: nombre de noeuds internes

\item

\begin{lstlisting}
DistMin
Input : un Arbre a
Procédé :
d <- 0
Pour chaque fils f dans a.Fils:
    df <- DistMin(f)
    si d = 0 ou si df < d:
        d <- df
Retourner d+1
\end{lstlisting}

\item

\begin{lstlisting}
DistMin
Input : un Arbre a
Procédé :
d <- 1
L <- [a]
Tant que Vrai:
    Lnext <- []
    Pour tout arbre a de L:
        Si a.nbFils = 0:
            Retourner d
        Pour tout fils f de a.fils:
            Ajouter f à Lnext
    d <- d+1
    L <- Lnext
\end{lstlisting}

\end{enumerate}


\end{exercice}