
\begin{exercice}[Complexité -- 5 pts].

\begin{enumerate}

\item Donner la complexité de chacun des algorithmes suivants (sans justification). Chaque algorithme 
prend en entrée un entier $n$. 

\begin{minipage}[t]{0.48 \textwidth}
\begin{lstlisting}
Algo_a :
    c <- 0
    Pour i allant de 1 à n:
        Pour j allant de 1 à i:
            Pour k allant de 1 à j:
                c <- c + 1
            
Algo_b :
    c <- 0
    Pour i allant de 1 à n:
        c <- c + 1
        
    Pour i allant de 1 à n:
       c <- c + 1     
\end{lstlisting}
\end{minipage}
\begin{minipage}[t]{0.48 \textwidth}
\begin{lstlisting}
Algo_c :
    c <- 0
    Tant que c < n:
        Algo_b(c)

Algo_d :
    c <- n
    Tant que c > 0:
        c <- c / 2 

Algo_e :
    c <- 1
    Tant que c < n:
        c <- c+2
\end{lstlisting}
\end{minipage}

\begin{lstlisting}
Algo_f :
    Si n = 0:
        Retourner 1
    Sinon :
        Retourner Algo_f(n-1) + Algo_f(n-1)
\end{lstlisting}

\item

Voici un algorithme récursif

\begin{lstlisting}
MonCalcul
Input :
    - n, un entier
Procédé :
    Si n = 0:
        Retourner 0
    Retourner 1 + MonCalcul(n/2)
\end{lstlisting}

\begin{enumerate}
\item Calculer la valeur retournée pour les entrées : 1, 2, 3, 4, 5.
\item Donner une valeur d'entrée telle que le résultat soit 4.
\item Que calcul cet algorithme ?
\item Quelle est sa complexité ?
\end{enumerate}


\end{enumerate}

\end{exercice}