\begin{exercice}[Structures de données]~

\begin{enumerate}
\item Quelle est la complexité de l'opération suivante : ajouter un élément dans un ensemble codé efficacement par table de hachage ?
\item Je cherche une structure telle que je puisse stocker une liste d'éléments et accéder à chacun d'eux en $O(1)$ par son index dans la liste. Que dois-je choisir ?
\item Une Pile est une structure qui possède deux opérations \emph{push} qui ajoute un élément et \emph{pop} qui supprime le dernier élément ajouté et le retourne. Si je code ma pile par une liste chaînée, ou dois-je ajouter / supprimer les éléments pour que la complexité des opérations \emph{push} et \emph{pop} soit $O(1)$ ? 
\end{enumerate}


\vspace{0.5cm}

\textbf{Solution}

\begin{enumerate}
\item $O(1)$
\item Tableau
\item Début de liste
\end{enumerate}


Question donnée au partiel 1 2018-2019, résultats obtenus :

\begin{tabular}{|c|c|c|c|}
\hline
A & C & D & E \\ \hline
$59\%$ & $36\%$ & $5\%$ & $0\%$ \\ \hline
\end{tabular} 




\end{exercice}