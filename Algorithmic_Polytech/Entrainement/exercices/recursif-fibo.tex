
\begin{exercice}[Partiel 2018-2019]
Voici un algorithme récursif

\begin{lstlisting}
MonCalcul
Input :
    - n, un entier
Procédé :
    Si n = 0 ou n = 1:
        Retourner 1
    Retourner MonCalcul(n-1) + MonCalcul(n-2)
\end{lstlisting}

\begin{enumerate}
\item Calculer la valeur retournée pour les entrées : 1, 2, 3, 4.
\item Sur quelles valeurs d'entrée cet algorithme termine-t-il ? (On ne demande pas les cas d'arrêt mais toutes les valeurs d'entrée où l'on obtient une réponse).
\item Combien d'appels récursifs sont effectués pour la valeur d'entrée 5.
\item Exprimer le nombre d'appels récursifs sous forme d'une fonction mathématique récursive.
\item Donner la complexité de la fonction.
\end{enumerate}

\textbf{Solution}

\begin{enumerate}
\item 1 -> 1 -- 2 -> 2 -- 3 -> 3 -- 4 -> 5
\item Termine si $n \geq 0$.
\item 15
\item $f(n) = f(n-1) + f(n-2)$
\item $O(2^n)$
\end{enumerate}

Question donnée au partiel 1 2018-2019, résultats obtenus :

\vspace{0.5cm}

\begin{tabular}{|c|c|c|c|c|}
\hline
A & B & C & D & E \\ \hline
$45\%$ & $14\%$ & $5\%$ & $36\%$ & $0\%$ \\ \hline
\end{tabular} 

\end{exercice}