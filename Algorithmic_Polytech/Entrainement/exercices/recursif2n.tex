
\begin{exercice}[Partiel 2017-18]
Voici un algorithme récursif

\begin{lstlisting}
MonCalcul
Input :
    - n, un entier
Procédé :
    Si n = 0:
        Retourner 1
    Retourner 2*MonCalcul(n-1)
\end{lstlisting}

\begin{enumerate}
\item Calculer la valeur retournée pour les entrées : 1, 2, 3.
\item Sur quelles valeurs d'entrée cet algorithme termine-t-il et que calcule-t-il de façon générale ?
\item Combien d'appels récursifs sont effectués pour la valeur d'entrée 10.
\item Exprimer le nombre d'appels récursives sous forme d'une fonction mathématiques récursives.
\item Donner la complexité de la fonction.
\end{enumerate}

\textbf{Solution}

\begin{enumerate}
\item 1 -> 2 -- 2 -> 4 -- 3 -> 8
\item Termine si $n \geq 0$ et calcule $2^n$.
\item 11 (ou 10 si on ne compte pas le premier appel)
\item $f(n) = f(n-1) + 1$ et $f(0) = 1$
\item $O(n)$
\end{enumerate}

Question donnée au partiel 1 2017-2018, résultats obtenus :

\vspace{0.5cm}

\begin{tabular}{|c|c|c|c|c|}
\hline
A & B & C & D & E \\ \hline
$42.9\%$ & $57.1\%$ & $0\%$ & $0\%$ & $0\%$ \\ \hline
\end{tabular} 
\end{exercice}