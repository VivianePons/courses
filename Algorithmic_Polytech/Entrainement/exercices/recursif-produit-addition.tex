
\begin{exercice}[2019 -- 2020]
Voici un algorithme récursif

\begin{lstlisting}
MonCalcul
Input :
    - a, b, 2 entiers
Procédé :
    Si a = 0 ou si b = 0:
        Renvoyer 0
    Si a < 0:
        Renvoyer MonCalcul(-a,b)
    Si a > b:
        Renvoyer MonCalcul(b,a)
    Si a%2 = 0:
        c <- MonCalcul(a/2,b)
        Renvoyer c+c
    Sinon :
        c <- MonCalcul((a-1)/2,b)
        Renvoyer b + c + c
\end{lstlisting}

\begin{enumerate}
\item Calculer la valeur retournée pour les entrées : $a = 8$ ,$b = 10$
\item Que calcule cet algorithme de façon générale ?
\item Sur quelles valeurs d'entrée cet algorithme termine-t-il ? (On ne demande pas les cas d'arrêt mais toutes les valeurs d'entrée où l'on obtient une réponse).
\item Combien d'appels récursifs sont effectués pour les valeurs $a = 8$ ,$b = 10$
\item Exprimer le nombre d'appels récursifs sous forme d'une fonction mathématique récursive.
\item Donner la complexité de la fonction.
\end{enumerate}

\begin{correction}
\begin{enumerate}
\item 80
\item $a \times b$
\item Termine pour tout $a,b$
\item 5
\item $f(a) = 1 + f(\frac{a}{2})$ 
\item $O(\log(n))$ où $n = min(a,b)$
\end{enumerate}
\end{correction}

Question donnée au partiel 2019 -- 2020, résultats obtenus :

\begin{tabular}{|c|c|c|c|c|}
\hline
A & B & C & D & E \\ \hline
41\% & 4\% & 33\% & 11\% & 11\% \\ \hline
\end{tabular} 

\end{exercice}