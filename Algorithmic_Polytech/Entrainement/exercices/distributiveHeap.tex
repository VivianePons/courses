
\begin{exercice}
On rappelle qu'un tas est un \textbf{arbre binaire parfait décroissant} :
\begin{itemize}
\item tous les niveaux sauf le dernier sont remplis au maximum et les feuilles du dernier niveaux sont alignées à gauche,
\item la valeur d'un nœud est toujours supérieure égale aux valeurs de ses fils.
\end{itemize}

La structure d'arbre binaire parfait permet au tas d'être stocké dans un tableau.

\begin{enumerate}
\item Donner le tableau correspondant au tas suivant

{ \newcommand{\nodea}{\node[draw,circle] (a) {$16$}
;}\newcommand{\nodeb}{\node[draw,circle] (b) {$11$}
;}\newcommand{\nodec}{\node[draw,circle] (c) {$10$}
;}\newcommand{\noded}{\node[draw,circle] (d) {$4$}
;}\newcommand{\nodee}{\node[draw,circle] (e) {$6$}
;}\newcommand{\nodef}{\node[draw,circle] (f) {$8$}
;}\newcommand{\nodeg}{\node[draw,circle] (g) {$5$}
;}\newcommand{\nodeh}{\node[draw,circle] (h) {$9$}
;}\newcommand{\nodei}{\node[draw,circle] (i) {$2$}
;}\newcommand{\nodej}{\node[draw,circle] (j) {$1$}
;}
\scalebox{0.8}{
\begin{tikzpicture}[auto]
\matrix[column sep=.3cm, row sep=.3cm,ampersand replacement=\&]{
         \&         \&         \&         \&         \&         \&         \& \nodea  \&         \&         \&         \\ 
         \&         \&         \& \nodeb  \&         \&         \&         \&         \&         \& \nodeh  \&         \\ 
         \& \nodec  \&         \&         \&         \& \nodef  \&         \&         \& \nodei  \&         \& \nodej  \\ 
 \noded  \&         \& \nodee  \&         \& \nodeg  \&         \&         \&         \&         \&         \&         \\
};

\path[ultra thick, red] (c) edge (d) edge (e)
	(f) edge (g)
	(b) edge (c) edge (f)
	(h) edge (i) edge (j)
	(a) edge (b) edge (h);
\end{tikzpicture}}
}

\item Dessiner l'arbre correspondant au tableau suivant 

\begin{tabular}{|c|c|c|c|c|c|c|c|c|c|}
\hline
\emph{indice} & 0 & 1 & 2 & 3 & 4 & 5 & 6 & 7 & 8 \\
\hline
\emph{valeur} & 5 & 4 & 2 & 3 & 2 & 1 & 1 & 2 & 1 \\
\hline
\end{tabular}
\end{enumerate}

On se place dans le contexte suivant : on possède un ensemble de serveurs qui gèrent des processus. Chaque serveur possède un certain nombre de \textit{cœurs} libres pour traiter de nouveaux processus. On stocke la disponibilité des serveurs dans un tableau sous forme de tas, par exemple

\begin{tabular}{|c|c|c|c|}
\hline
 0 & 1 & 2 & 3 \\
\hline
4 & 2 & 2 & 0 \\
\hline
\end{tabular}

signifie que j'ai 4 serveurs. Le serveur 0 possède 4 cœurs disponibles, les serveurs 1 et 2 en possèdent chacun 2 et le serveur 3, 0. Le serveur 0 sera toujours celui avec le plus de cœurs disponibles.


\begin{enumerate}
\setcounter{enumi}{2}

\item Considérer les 3 exemples suivants :

\begin{tabular}{ccc}
{ \newcommand{\nodea}{\node[draw,circle] (a) {4}
;}\newcommand{\nodeb}{\node[draw,circle] (b) {4}
;}\newcommand{\nodec}{\node[draw,circle] (c) {2}
;}\newcommand{\noded}{\node[draw,circle] (d) {3}
;}\newcommand{\nodee}{\node[draw,circle] (e) {3}
;}\newcommand{\nodef}{\node[draw,circle] (f) {2}
;}\newcommand{\nodeg}{\node[draw,circle] (g) {2}
;}
\scalebox{0.6}{
\begin{tikzpicture}[auto]
\matrix[column sep=.3cm, row sep=.3cm,ampersand replacement=\&]{
         \&         \&         \& \nodea  \&         \&         \&         \\ 
         \& \nodeb  \&         \&         \&         \& \nodee  \&         \\ 
 \nodec  \&         \& \noded  \&         \& \nodef  \&         \& \nodeg  \\
};

\path[ultra thick, red] (b) edge (c) edge (d)
	(e) edge (f) edge (g)
	(a) edge (b) edge (e);
\end{tikzpicture}}
}
&
{ \newcommand{\nodea}{\node[draw,circle] (a) {3}
;}\newcommand{\nodeb}{\node[draw,circle] (b) {2}
;}\newcommand{\nodec}{\node[draw,circle] (c) {2}
;}\newcommand{\noded}{\node[draw,circle] (d) {1}
;}\newcommand{\nodee}{\node[draw,circle] (e) {3}
;}\newcommand{\nodef}{\node[draw,circle] (f) {0}
;}\newcommand{\nodeg}{\node[draw,circle] (g) {0}
;}
\scalebox{0.6}{
\begin{tikzpicture}[auto]
\matrix[column sep=.3cm, row sep=.3cm,ampersand replacement=\&]{
         \&         \&         \& \nodea  \&         \&         \&         \\ 
         \& \nodeb  \&         \&         \&         \& \nodee  \&         \\ 
 \nodec  \&         \& \noded  \&         \& \nodef  \&         \& \nodeg  \\
};

\path[ultra thick, red] (b) edge (c) edge (d)
	(e) edge (f) edge (g)
	(a) edge (b) edge (e);
\end{tikzpicture}}
}
&
{ \newcommand{\nodea}{\node[draw,circle] (a) {4}
;}\newcommand{\nodeb}{\node[draw,circle] (b) {4}
;}\newcommand{\nodec}{\node[draw,circle] (c) {2}
;}\newcommand{\noded}{\node[draw,circle] (d) {4}
;}\newcommand{\nodee}{\node[draw,circle] (e) {3}
;}\newcommand{\nodef}{\node[draw,circle] (f) {2}
;}\newcommand{\nodeg}{\node[draw,circle] (g) {2}
;}
\scalebox{0.6}{
\begin{tikzpicture}[auto]
\matrix[column sep=.3cm, row sep=.3cm,ampersand replacement=\&]{
         \&         \&         \& \nodea  \&         \&         \&         \\ 
         \& \nodeb  \&         \&         \&         \& \nodee  \&         \\ 
 \nodec  \&         \& \noded  \&         \& \nodef  \&         \& \nodeg  \\
};

\path[ultra thick, red] (b) edge (c) edge (d)
	(e) edge (f) edge (g)
	(a) edge (b) edge (e);
\end{tikzpicture}}
}
\end{tabular}

Pour chacun des exemples, 
\begin{enumerate}
\item Diminuer la valeur de {\tt T[0]} de 1 (on lui affecte un processus) et regarder si le résultat obtenu est un tas.
\item Si ce n'est pas le cas, décrire le processus à suivre pour \emph{corriger} le résultat et obtenir un tas.
\end{enumerate}

\item Implanter une fonction qui réalise de façon générale l'algorithme décrit sur les exemples précédents. Quelle est sa complexité ?

\end{enumerate}

\textbf{Solution}

\begin{enumerate}
\item

\begin{tabular}{|c|c|c|c|c|c|c|c|c|c|}
\hline
 0 & 1 & 2 & 3 & 4 & 5 & 6 & 7 & 8 & 9\\
\hline
16 & 11 & 9 & 10 & 8 & 2 & 1 & 4 & 6 & 5 \\
\hline
\end{tabular}

\item

{ \newcommand{\nodea}{\node[draw,circle] (a) {$5$}
;}\newcommand{\nodeb}{\node[draw,circle] (b) {$4$}
;}\newcommand{\nodec}{\node[draw,circle] (c) {$3$}
;}\newcommand{\noded}{\node[draw,circle] (d) {$2$}
;}\newcommand{\nodee}{\node[draw,circle] (e) {$1$}
;}\newcommand{\nodef}{\node[draw,circle] (f) {$2$}
;}\newcommand{\nodeh}{\node[draw,circle] (h) {$2$}
;}\newcommand{\nodei}{\node[draw,circle] (i) {$1$}
;}\newcommand{\nodej}{\node[draw,circle] (j) {$1$}
;}
\scalebox{0.8}{
\begin{tikzpicture}[auto]
\matrix[column sep=.3cm, row sep=.3cm,ampersand replacement=\&]{
         \&         \&         \&         \&         \&         \&         \& \nodea  \&         \&         \&         \\ 
         \&         \&         \& \nodeb  \&         \&         \&         \&         \&         \& \nodeh  \&         \\ 
         \& \nodec  \&         \&         \&         \& \nodef  \&         \&         \& \nodei  \&         \& \nodej  \\ 
 \noded  \&         \& \nodee  \&         \&   \&         \&         \&         \&         \&         \&         \\
};

\path[ultra thick, red] (c) edge (d) edge (e)
	(b) edge (c) edge (f)
	(h) edge (i) edge (j)
	(a) edge (b) edge (h);
\end{tikzpicture}}
}

\item Aucun des trois ne reste un tas : il faut échanger la valeur avec son fils maximal puis recommencer récursivement sur le fils échangé.

\item Une solution possible

\begin{lstlisting}
Input : Tableau T
Procédé :
    i <- 0
    Tant que 2*i+1 < T.taille:
       filsMax <- 2*i+1
       Si 2*i +2 < T.taille et T[2*i+2] > T[2*i+1]:
           filsMax <- 2*i+2
       Si T[i] < T[filsMax]:
           Echanger T[i], T[filsMax]
           i <- filsMax    
       Sinon
           break
\end{lstlisting}
\end{enumerate}
\end{exercice}