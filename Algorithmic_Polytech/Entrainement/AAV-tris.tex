\documentclass{../cours}
\usepackage{hyperref}
\title{Entraînement : algorithmes de tris}

\begin{document}
\maketitle

Pour tous les exercices, la grille d'évaluation est la suivante.

\subsection*{Identification de l'algorithme et complexité}~

\begin{tabular}{|l|p{12cm}|}
\hline
A (20) & \small{Algorithme bien identifié / décrit et bonne complexité} \\ \hline
C (11) & \small{Complexité correcte mais algorithme mal identifié / décrit} \\ \hline
D (8) & \small{Algorithme bien identifié / décrit mais erreur dans la complexité} \\ \hline
E (1) & \small{Ni complexité, ni identification juste} \\ \hline
\end{tabular}

\subsection*{Réalisation de la partie d'algorithme manquante}~

\begin{tabular}{|l|p{12cm}|}
\hline
A (20) & \small{L'algorithme répond correctement au problème posé, il est écrit de façon claire et compacte et est de complexité optimale.} \\ \hline
B (16) & \small{L'algorithme contient quelques erreurs mais reste globalement juste et la complexité optimale est respectée.} \\ \hline
C (11) & \small{L'algorithme fonctionne globalement mais complexité non optimale.} \\ \hline
D (8) & \small{L'algorithme ne fonctionne pas.} \\ \hline
E (1) & \small{Algorithme quasi inexistant ou ne répondant pas du tout au problème posé.} \\ \hline
\end{tabular}



\begin{exercice}[Tri rapide -- 5 pts]

~
\begin{enumerate}
\item Rappeler en quelques phrases le principe du tri rapide (quicksort).
\end{enumerate}

En voici une implantation partielle :

\begin{lstlisting}
TriRapide
Input :
    - T, un tableau de nombres
    - a, le premier indice de la zone à trier
    - b, le dernier indice de la zone à trier 
Procédé :
    Si b <= a:
        Retourner
    m <- Pivot(T,a,b)
    TriRapide(T,a,m-1)
    TriRapide(T,m+1,b)
\end{lstlisting}

Pour un tableau {\tt T} de taille 5, on appellerait la fonction de cette façon {\tt TriRapide(T,0,4)} car 4 est le dernier indice du tableau.

La fonction {\tt Pivot} modifie le tableau de telle sorte qu'après son passage, toutes les valeurs avant l'indice {\tt m} doivent être plus petite ou égale à {\tt T[m]} et toutes les valeurs après l'indice {\tt m} doivent être plus grande que {\tt T[m]}. Par exemple, elle pourrait modifier le tableau suivant de cette façon :

Avant le passage de Pivot :

\begin{tabular}{ccccc}
5 & 2 & 6 & 3 & 1
\end{tabular}

Après le passage de Pivot :

\begin{tabular}{ccccc}
3 & 2 & 1 & \red{5} & 6
\end{tabular}

L'indice {\tt m} retourné par la fonction dans ce cas est \textbf{3}.

\begin{enumerate}
\setcounter{enumi}{1}

\item Donner une implantation de la fonction {\tt Pivot}

\textit{Remarque : } votre fonction n'est pas obligée d'agir exactement comme dans l'exemple, elle doit seulement respecter les propriétés énoncées ci-dessus.

\item Donner les étapes de votre propre algorithme de pivot sur le tableau donné en exemple:

\begin{tabular}{ccccc}
5 & 2 & 6 & 3 & 1
\end{tabular}

\item Quelle est la complexité de l'algorithme {\tt TriRapide} dans les trois cas suivant : tableau déjà trié, meilleur des cas, en moyenne.
\end{enumerate}

\textbf{Solution}

\begin{enumerate}
\item Principe : A chaque étape on choisit un pivot $v$ (par exemple, le premier élément du tableau), puis on place les éléments du tableau tel que tous les éléments \emph{plus petits} que $v$ soient sur la gauche du tableau, tous les éléments \emph{plus grands} sur la droite et $v$ entre la partie gauche et la partie droite. Puis on trie récursivement les parties gauches et droites.

\item cf cours
\item cf cours
\item tableau déjà trié : $O(n^2)$, meilleur des cas : $O(n \log(n))$, en moyenne $O(n \log(n))$.
\end{enumerate}

\end{exercice}


\begin{exercice}[Tri Fusion -- 5 pts]

~
\begin{enumerate}
\item Rappeler en quelques phrases le principe du tri fusion.
\end{enumerate}

En voici une implantation partielle :

\begin{lstlisting}
TriFusion
Input :
    - T, un tableau de taille n 
Output :
    - Un tableau trié contenant les mêmes valeurs que T 
Procédé :
    Si n <= 1:
        Retourner T
    m <- n/2
    T1 <- TriFusion(T[:m+1])
    T2 <- TriFusion(T[m+1:])
    Retourner Fusion(T1,T2)
    
    
Fusion
Input :
    - T1, un tableau de taille m1 supposé trié
    - T2, un tableau de taille m2 supposé trié
Output:
    - un tableau trié contenant les valeurs de T1 et T2
Procédé
    ....
\end{lstlisting}

L'écriture {\tt T[:m+1]} signifie une copie du tableau {\tt T} entre les indices 0 et $m$ inclus, {\tt T[m+1:]} signifie une copie du tableau {\tt T} entre les indices $m+1$ et $n-1$ inclus.

Voici un exemple du résultat de la fonction {\tt Fusion} :


\begin{tabular}{cccccc}
{\tt T1} 2 & 2 & 4 & 5 & 5
\end{tabular}

\begin{tabular}{ccccc}
{\tt T2} 3 & 4 & 6 & 7
\end{tabular}

la fonction renvoie

\begin{tabular}{ccccccccc}
2 & 2 & 3 & 4 & 4 & 5 & 5 & 6 & 7
\end{tabular}

\begin{enumerate}
\setcounter{enumi}{1}

\item Compléter le procédé de la fonction {\tt Fusion}. Votre algorithme doit avoir une complexité $O(n)$ où $n = m1 + m2$.

\item Donner les étapes de votre propre algorithme de fusion sur l'exemple ci-dessus.

\item Quelle est la complexité de l'algorithme complet {\tt TriFusion} dans le pire des cas ?
\end{enumerate}

\textbf{Solution}

\begin{enumerate}
\item L'algorithme découpe le tableau en deux parts égales et trie les deux sous-tableaux puis les fusionne en un tableau unique.
\item cf cours
\item cf cours
\item Pire des cas : $O(n \log(n))$
\end{enumerate}

\end{exercice}


\end{document}
